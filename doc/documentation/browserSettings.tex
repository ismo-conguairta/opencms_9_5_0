\chapter{Client Setup}
\label{Client Setup}

\textbf{All configurations issues on this page only apply when you
set up a workstation client computer to access the backoffice part of OpenCms.}

The backoffice is the part where you can edit pages, create new pages,
manage users etc. In OpenCms this is called "the Workplace" and it
makes use of dynamic HTML and some plugins to provide editing
functionality that can not be achieved using standard HTML.

\textbf{The HTML generated by OpenCms as output for the final
website is 100\% controlled by you and no extra setup or
plugin is needed for clients to access the OpenCms generated sites.}

For the development of OpenCms, we so far are focused on the
implementation of the server components. To save time during the
development we decided to use existing components for specific
client functions. This means that ActiveX components were used for
the user editors to offer the end user a rich set of
functionality. This, however, currently restricts the WYSIWYG
editor to Microsoft Internet Explorer. For Netscape Navigator, a
simple Textarea is used as source code editor for both WYSIWYG and
source code mode.

It would be possible to replace the ActiveX controls with Java
applets or other technology. Recently some Java applets have become available
in the open source space that could be fitted into OpenCms. 
We would appreciate any contribution of such an alternative editor.
If you are interested in doing this, please send an email 
to {\em contributions@opencms.org}


\section{Configuring MS Internet Explorer 5.5 and 6.x Clients}
\label{browsersettings}

First step is the installation of the neccessary controls. 
The following components are needed by OpenCms:

\begin{enumerate}
\item For the WYSIWYG editor, the "Dynamic HTML Edit Control" is
used. This control is part of all MS IE installations since
version 5.0, which means that it is already installed if your IE version 
is not older then 5.0. 

\item The source code editor is a component developed by 
    \rqhttp{http://www.aysoft.com}{AY Software} and it's
called LeEdit OCX Control. You can download the shareware version
of this control from the site:\\
\rqhttp{http://www.aysoft.com/ledit.htm}{http://www.aysoft.com/ledit.htm}.
This control must be installed on all clients that need access to
the code editor functionality.
If you do not install this component, OpenCms will provide a html
textarea for the source code input, which is less convenient but
also usable in general.
\end{enumerate}

\texttt{Note}: The next step is optional and only required for the full functionality of the source code editor.
If you don't need the advanced functions like search, replace or undo in the source code editing mode 
and want to work with a html textarea, 
you can ignore the following ActiveX settings.

The second step is configuring the ActiveX settings for the source code editor to work properly.
Open IEs "Internet options." Then do the
following:

\begin{enumerate}
\item On the tab "Security", select "Trusted site zones" from the
drop-down menu and click on "Add Sites" to add the URL (e.g.
http://opencms.mycompany.com - ask your system administrator for
the exact URL) of the zone's OpenCms server. Deactivate the radio
button "Require server verification (https:) for all sites in this
zone."
\item On the tab "Security", select "Trusted site zones"
from the drop-down menu and click on "Settings". All ActiveX
control elements must be set to "Enable." A note on security: It
is safe to use ActiveX controls with these settings since their
use is allowed only for the "Trusted sites", and ActiveX remains
disabled for all other web sites.
\end{enumerate}

This setup must be repeated for all clients / workstations that
use the OpenCms workplace. Cookies and JavaScript must be enabled 
on these machines.


\section{Configuring Netscape Navigator 7.x Clients}

In Netscape Navigator OpenCms switches automatically to an html 
text area in which the end user can edit the content as html 
source code. A WYSIWYG editor is currently not available for
Netscape Navigator Clients.

Cookies and JavaScript must be enabled to use the workplace.
