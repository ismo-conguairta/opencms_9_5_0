\chapter{About this documentation}
\label{About this documentation}

\section{Version of this document}

This document is \verb Revision: 1.9 \\
It was last changed as of \verb 2005/06/28 \\

\section{The documentation structure of OpenCms 5.0}

We have finally decided to eat our own dogfood and manage the contents of the documentation 
with our content management system.
So from version 5.0 onward, the main part of the documentation will be maintained 
in the form of OpenCms interactive documentation modules.
These are content modules that you can install and read directly in OpenCms.
One of the big advantages is that this gives the reader an interactive documentation 
with a lot of working example code,
because everything can directly be checked out in the running system.
It's also easier to write the documentation that way.

The main purpose of this book is now to provide introductory material about OpenCms,
especially providing installation instructions, an ``end user" introduction and a general system overview.

It will take some time to shift all of the contents of the previous ``book only" version to 
the content module form, so this document will likely be around for some time.
However, the new functionality of 5.0 and the following releases are not described here,
but only in the content modules. 
Please refer to chapter \ref{About the archived documentation} for more information about 
the chapters of the documentation that still have to be shifted to the interactive format.

Of course, if someone wants to contribute his time and do the work to 
put the interactive documentation modules back into this ``static" form, this is very much appreciated. 
On the other hand we would also appreciate if someone would make interactive documentation modules
from the parts of the book that are still described only in this static document.
Please send an email to {\em contributions@opencms.org} if you are interested in doing any of this.

\section{Where to get the documentation modules}

As of the writing of this document, the following documentation modules are available:

\subsection{OpenCms documentation base}
This module provides a set of JSP templates and elements that are used by
all other documentation modules provided by Alkacon Software GmbH. {\em This module is required
by all other {\tt com.alkacon.*} documentation modules listed here and has to be installed first!}
\\
\\
\begin{tabular}{ll}
Package name: & {\tt com.alkacon.documentation}\\
Available from: & {\tt http://www.alkacon.com/downloads/}\\
Provided by: & {\em Alkacon Software GmbH}\\
\end{tabular}

\subsection{OpenCms JSP basic documentation}
A basic introduction to JSP in OpenCms: How to create, edit \& publish a JSP, options to access 
the OpenCms system and some technical background information.
\\
\\
\begin{tabular}{ll}
Package name: & {\tt com.alkacon.documentation.documentation-jsp}\\
Depends on: & {\tt com.alkacon.documentation}\\
Available from: & {\tt http://www.alkacon.com/downloads/}\\
Provided by: & {\em Alkacon Software GmbH}\\
\end{tabular}

\subsection{OpenCms JSP taglib documentation}
OpenCms includes it's own JSP taglib. This is the reference documentation (with test cases) for 
each tag of the OpenCms Taglib.
\\
\\
\begin{tabular}{ll}
Package name: & {\tt com.alkacon.documentation.documentation-taglib}\\
Depends on: & {\tt com.alkacon.documentation}\\
Available from: & {\tt http://www.alkacon.com/downloads/}\\
Provided by: & {\em Alkacon Software GmbH}\\
\end{tabular}

\subsection{OpenCms JSP scriptlet documentation}
Learn here how to use the same functionality offered by the OpenCms JSP taglib in JSP scriptlet 
code in your JSPs.
\\
\\
\begin{tabular}{ll}
Package name: & {\tt com.alkacon.documentation.documentation-scriptlet}\\
Depends on: & {\tt com.alkacon.documentation}\\
Available from: & {\tt http://www.alkacon.com/downloads/}\\
Provided by: & {\em Alkacon Software GmbH}\\
\end{tabular}

\subsection{OpenCms Flexcache documentation}
The Flexcache can cache JSP output to increase the speed of page delivery. This is the complete 
documentation of all FlexCache configuration options and parameters to set up your page caching 
policy using the Flexcache.
\\
\\
\begin{tabular}{ll}
Package name: & {\tt com.alkacon.documentation.documentation-flexcache}\\
Depends on: & {\tt com.alkacon.documentation}\\
Available from: & {\tt http://www.alkacon.com/downloads/}\\
Provided by: & {\em Alkacon Software GmbH}\\
\end{tabular}

\subsection{OpenCms module documentation}
Get acquainted with the OpenCms module mechanism that allows you to easily bundle and distribute 
templates and JSPs.
\\
\\
\begin{tabular}{ll}
Package name: & {\tt com.alkacon.documentation.documentation-modules}\\
Depends on: & {\tt com.alkacon.documentation}\\
Available from: & {\tt http://www.alkacon.com/downloads/}\\
Provided by: & {\em Alkacon Software GmbH}\\
\end{tabular}

\subsection{Flex examples set 1}
The first set of tests and examples developed for the OpenCms / JSP Flex integration package. 
Basic stuff dealing with inclusion, caching, headers and redirects.
\\
\\
\begin{tabular}{ll}
Package name: & {\tt com.alkacon.documentation.examples-flex-1}\\
Depends on: & {\tt com.alkacon.documentation}\\
Available from: & {\tt http://www.alkacon.com/downloads/}\\
Provided by: & {\em Alkacon Software GmbH}\\
\end{tabular}

\subsection{Flex examples set 2}
The second set of tests and examples, including demos on how to build a simple template from JSP.
\\
\\
\begin{tabular}{ll}
Package name: & {\tt com.alkacon.documentation.examples-flex-2}\\
Depends on: & {\tt com.alkacon.documentation}\\
Available from: & {\tt http://www.alkacon.com/downloads/}\\
Provided by: & {\em Alkacon Software GmbH}\\
\end{tabular}

\subsection{Flex examples set 3}
Further test cases and examples for the OpenCms/Flex JSP integration. These are rather 
specialized as they deal with encoding and inclusion of subelements.
\\
\\
\begin{tabular}{ll}
Package name: & {\tt com.alkacon.documentation.examples-flex-3}\\
Depends on: & {\tt com.alkacon.documentation}\\
Available from: & {\tt http://www.alkacon.com/downloads/}\\
Provided by: & {\em Alkacon Software GmbH}\\
\end{tabular}

\subsection{Original JSTL 1.0 standard taglib examples}
The JSTL will be part of the next JSP/Servlet specification release. These are the original 
JSTL 1.0 examples imported into OpenCms.
\\
\\
\begin{tabular}{ll}
Package name: & {\tt com.alkacon.documentation.examples-jstl}\\
Depends on: & {\tt com.alkacon.documentation}\\
Available from: & {\tt http://www.alkacon.com/downloads/}\\
Provided by: & {\em Alkacon Software GmbH}\\
\end{tabular}

\subsection{Original Tomcat 4.x JSP examples}
The original Tomcat 4.x JSP examples imported into OpenCms. Check these out to see what you can 
do with JSP in OpenCms.
\\
\\
\begin{tabular}{ll}
Package name: & {\tt com.alkacon.documentation.examples-tomcat}\\
Depends on: & {\tt com.alkacon.documentation}\\
Available from: & {\tt http://www.alkacon.com/downloads/}\\
Provided by: & {\em Alkacon Software GmbH}\\
\end{tabular}

\subsection{Howto: JSP template development}
An introduction in the JSP template development with OpenCms. Learn how to create templates 
for your web site.
\\
\\
\begin{tabular}{ll}
Package name: & {\tt com.alkacon.documentation.howto-template}\\
Depends on: & {\tt com.alkacon.documentation}\\
Available from: & {\tt http://www.alkacon.com/downloads/}\\
Provided by: & {\em Alkacon Software GmbH}\\
\end{tabular}

\subsection{Howto: Translating the OpenCms workplace}
A detailed description how to translate the OpenCms workplace and the online help to another language.
\\
\\
\begin{tabular}{ll}
Package name: & {\tt com.alkacon.documentation.howto-workplace-translation}\\
Depends on: & {\tt com.alkacon.documentation}\\
Available from: & {\tt http://www.alkacon.com/downloads/}\\
Provided by: & {\em Alkacon Software GmbH}\\
\end{tabular}